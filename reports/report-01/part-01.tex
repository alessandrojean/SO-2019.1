\question
Uso da chamada de sistema \verb|fork|.

\begin{parts}
  \part
  Compile e execute o programa \verb|prog1.c|. Explique o seu
  funcionamento, detalhando o uso das chamadas de sistema
  usadas.

  \begin{solution}
    O programa cria uma cópia de si mesmo através da chamada de
    \verb|fork|, obtendo como retorno um \verb|PID| para diferenciar
    o processo pai do filho. O processo pai espera a execução do
    filho terminar, através da chamada de \verb|waitpid|, obtendo
    seu \emph{status}. Assim, é possível diferenciar se o filho
    foi terminado normalmente, ou teve algum erro, ou foi morto
    através de alguma chamada de \verb|kill|.

    \begin{Verbatim}[label={\$ ./prog1}]
    Pai com filho com PID 10111.
    Aguardando término do filho!
    Filho (PID 10111) vai dormir por 10 segundos.
    Filho terminando a execução.
    Pai viu filho com PID 10111 terminar.
    Filho terminou com status = 0.
    Pai terminando a execução.
    \end{Verbatim}
  \end{solution}

  \part
  Envie o sinal \verb|SIGINT| ao processo filho: 
  \verb|kill -SIGINT PID|, em que \verb|PID| é o PID do processo
  filho mostrado na tela. O que aconteceu nesse caso? Justifique.

  \begin{solution}
    Neste caso o processo filho foi terminado antes de seu término
    de forma voluntária, sendo encerrado com sinal \verb|2|, o
    \verb|SIGINT|, sendo equivalente ao \verb|Ctrl + C| em um terminal.

    \begin{Verbatim}[label={\$ ./prog1}]
    Pai com filho com PID 10250.
    Aguardando término do filho!
    Filho (PID 10250) vai dormir por 10 segundos.
    Pai viu filho com PID 10250 terminar.
    Filho foi morto por sinal 2.
    Pai terminando a execução.
    \end{Verbatim}
  \end{solution}
\end{parts}

\question
Sinais como condições de erro.

\begin{parts}
  \part
  Compile e execute o programa \verb|prog2.c|. Por que o processo filho
  é terminado?

  \begin{solution}
    O processo filho é terminado por causa de uma operação de divisão
    por zero, forçando o programa a encerrar sua execução com um
    sinal \verb|8|, o \verb|SIGFPE|.

    \begin{Verbatim}[label={\$ ./prog2}]
    Pai com filho com PID 10421.
    Aguardando término do filho!
    Pai viu filho com PID 10421 terminar.
    Filho foi morto por sinal 8.
    Pai terminando a execução.
    \end{Verbatim}
  \end{solution}

  \part
  Compile e execute o programa \verb|prog3.c|. Por que o processo filho
  é terminado?

  \begin{solution}
    O processo filho é terminado por causa de um acesso a um endereço
    de memória inválido, forçando o programa a encerrar sua execução
    com o sinal \verb|11|, o \verb|SIGSEGV|.
  
    \begin{Verbatim}[label={\$ ./prog3}]
    Pai com filho com PID 10490.
    Aguardando término do filho!
    Pai viu filho com PID 10490 terminar.
    Filho foi morto por sinal 11.
    Pai terminando a execução.
    \end{Verbatim}
  \end{solution}
\end{parts}

\question
Como tratar um sinal?

\begin{parts}
  \part
  Compile e execute o programa \verb|prog4.c|. Em uma segunda execução,
  envie o sinal \verb|SIGINT| ao processo filho. Compare as duas execuções
  e comente. O que é necessário fazer para tratar um sinal?

  \begin{solution}
    Como o programa possui um \emph{handler} para quando receber o
    \verb|SIGINT|, quando o mesmo receber este sinal através de uma
    chamada de \verb|kill|, o mesmo tratará este sinal com a função
    \emph{handler}, não sendo interrompido e terminando com 
    \emph{status} \verb|0|.

    Para tratar um sinal é preciso criar uma função de \emph{handler}
    e informá-la ao sistema de forma explicita atrtavés da chamada
    de \verb|signal|, informando qual sinal deseja tratar e qual
    a função \emph{handler}.

    \begin{Verbatim}[label={\$ ./prog4}]
    Pai com filho com PID 10691.
    Aguardando término do filho!
    Filho (PID 10691) vai dormir por 10 segundos.
    Filho terminando a execução.
    Pai viu filho com PID 10691 terminar.
    Filho terminou com status = 0.
    Pai terminando a execução.
    \end{Verbatim}

    \begin{Verbatim}[label={\$ ./prog4}]
    Pai com filho com PID 10668.
    Aguardando término do filho!
    Filho (PID 10668) vai dormir por 10 segundos.
    Tratando sinal SIGINT do processo 10668!
    Na verdade, acho que não vou fazer nada! :-) 
    Filho terminando a execução.
    Pai viu filho com PID 10668 terminar.
    Filho terminou com status = 0.
    Pai terminando a execução.
    \end{Verbatim}
  \end{solution}

  \part
  Modifique o programa \verb|prog4.c| para tratar a condição de erro
  do programa \verb|prog3.c|. Foi possível tratar o sinal e continuar
  a execução? Justifique.

  \begin{solution}
    Não é possível tratar uma função de erro que necessariamente deve
    forçar o programa a ser terminado. O sistema informa o programa
    sobre o erro, o programa trata-o, no entanto o problema ainda
    permanece, fazendo com que o programa fique em um \emph{loop}
    infinito. Para evitar tal comportamento, pode-se encerrar o
    programa na função de \emph{handler}, com, por exemplo, o
    \verb|exit(13)|, gerando um \emph{status} de erro que pode
    ser usado futuramente para \emph{debug}.

    \begin{Verbatim}[label={\$ ./prog4-e}]
    Pai com filho com PID 10958.
    Aguardando término do filho!
    Tratando sinal SIGSEGV do processo 10958!
    Pai viu filho com PID 10958 terminar.
    Filho terminou com status = 13.
    Pai terminando a execução.
    \end{Verbatim}
  \end{solution}
\end{parts}

\question
Como mascarar um sinal?

\begin{parts}
  \part
  Compile e execute o programa \verb|prog5.c|. Durante sua execução, envie
  o sinal \verb|SIGQUIT| ao processo filho. Em uma segunda execução, envie
  o sinal \verb|SIGTERM| ao processo filho. Compare as duas execuções e
  comente. O que é necessário fazer para mascarar um sinal?

  \begin{solution}
    A primeira execução não é terminada pois o mesmo possui uma máscara
    para o sinal \verb|SIGQUIT|, ignorando solenimente o sinal, continuando
    sua execução de uma forma normal. A segunda execução é terminada
    pois o mesmo não possui uma máscara ou tratamento para o sinal
    \verb|SIGTERM|, forçando o término do processo.

    Para mascarar um sinal é necessário criar uma máscara e instalá-la
    para o processo. Isto é feito através das chamadas a \verb|sigemptyset|,
    \verb|sigaddset| e \verb|sigprocmask|.

    \begin{Verbatim}[label={\$ ./prog5}]
    Pai com filho com PID 11046.
    Aguardando término do filho!
    Filho (PID 11046) vai dormir por 10 segundos.
    Filho terminando a execução.
    Pai viu filho com PID 11046 terminar.
    Filho terminou com status = 0.
    Pai terminando a execução.
    \end{Verbatim}

    \begin{Verbatim}[label={\$ ./prog5}]
    Pai com filho com PID 11035.
    Aguardando término do filho!
    Filho (PID 11035) vai dormir por 10 segundos.
    Pai viu filho com PID 11035 terminar.
    Filho foi morto por sinal 15.
    Pai terminando a execução.
    \end{Verbatim}
  \end{solution}

  \part
  É possível tratar ou mascarar o sinal \verb|SIGKILL| ou \verb|SIGSTOP|?
  Justifique.

  \begin{solution}
    Os sinais \verb|SIGKILL| e \verb|SIGSTOP| não podem ser tratados
    nem mascarados pois deve existir a opção de matar ou pausar o processo,
    respectivamente.
  \end{solution}
\end{parts}

\question
Criação de threads.

\begin{parts}
  \part
  Compile com a opção \verb|-pthread| o programa \verb|prog6.c|. Execute-o
  várias vezes e comente sobre as saídas. Explique o funcionamento 
  do programa.

  \begin{solution}
    O programa cria 10 threads através de \verb|pthread_create|, esperando
    que todos executem sua função e espera que todos terminem através
    de \verb|pthread_join| para encerrar o programa. Cada execução do
    programa vai ter uma ordem diferente de execução já que os threads
    são escalonados de maneira independente pelo núcleo do sistema
    operacional.
    
    \begin{Verbatim}[label={\$ ./prog6}]
    Thread 2 executou.
    Thread 1 executou.
    Thread 0 executou.
    Thread 3 executou.
    Thread 4 executou.
    Thread 5 executou.
    Thread 6 executou.
    Thread 8 executou.
    Thread 9 executou.
    Thread 7 executou.
    Threads terminaram. Fim do programa.
    \end{Verbatim}
  \end{solution}

  \part
  Compile com a opção \verb|-pthread| o programa \verb|prog7.c|. 
  Execute-o várias vezes e comente sobre as saídas. Qual a solução
  para o problema apresentado?

  \begin{solution}
    Entre a atribuição e o valor da variável há uma decisão de 
    escalonamento, além do fato do programa ter uma área de memória
    compartilhada, gerando condições de corrida. A cada execução,
    dependendo de quem e quando é executado, temos um resultado
    diferente. Para solucionar este problema apresentado é preciso
    implementar um mecanismo de exclusão mútua que limita somente
    um thread a cada momento na região crítica.

    \begin{Verbatim}[label={\$ ./prog7}]
    Thread 0, s = 1.
    Thread 1, s = 1.
    \end{Verbatim}
  \end{solution}
\end{parts}

\question
Implementando a exclusão mútua.

\begin{parts}
  \part
  Compile com a opção \verb|-pthread| o programa \verb|prog8.c|.
  Execute-o várias vezes e comente sobre as saídas.

  \begin{solution}
    O programa protege a região crítica das duas threads, impedindo
    com que ambas as threads executem ao mesmo tempo na região de 
    memória compartilhada. Desta maneira, independentemente da decisão
    de escalonamento, não há mais condições de corrida e a variável
    \verb|s| sempre vai ter o valor esperado na impressão.

    \begin{Verbatim}[label={\$ ./prog8}]
    Thread 0, s = 0.
    Thread 1, s = 1.
    \end{Verbatim}
  \end{solution}
\end{parts}

\question
Produtor-consumidor com troca de mensagens.

\begin{parts}
  \part
  Compile com a opção \verb|-o cons| o programa \verb|prog9c.c|.
  Compile com a opção \verb|-o prod| o programa \verb|prog9p.c|.
  Execute os programas \verb|cons| e \verb|prod| em diversas
  combinações. Explique o funcionamento dos programas.

  \begin{solution}
    O produtor coloca uma mensagem na caixa-postal e o consumidor
    retira esta mensagem. Caso a caixa-postal esteja vazia, o 
    consumidor fica travado esperando por uma mensagem.

    \begin{Verbatim}[label={\$ ./cons}]
    Erro na criacao de mailbox: Success
    o tipo da mensagem eh: 1
    o conteudo da mensagem eh: Esta eh uma mensagem de teste
    \end{Verbatim}

    \begin{Verbatim}[label={\$ ./prod}]
    erro na criacao de mailbox: Success
    o resultado eh: 0
    \end{Verbatim}
  \end{solution}

  \part
  Modifique os programas \verb|prog9c.c| e \verb|prog9p.c| para
  implementar o sistema produtor consumidor com $N$ mensagens
  da Figura 2.29 do livro texto (Aula 3 - página 30), mas fazendo
  uso de caixa postal.

  \begin{solution}
    As modificações efetuadas incluem a criação de mais uma caixa postal
    para uma comunicação no sentido inverso, ou seja, consumidor-produtor,
    para que o consumidor possa informar que deseja $N$ mensagens ao
    produtor, para que assim o produtor as envie. A modificação está
    disponível juntamente ao arquivo compactado onde este relatório
    encontra-se originalmente.
  \end{solution}
\end{parts}