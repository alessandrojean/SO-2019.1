\question
Política de escalonamento e prioridades.

\begin{parts}
  \part
  Compile com a opção \verb|-pthread| e execute o programa \verb|prog1.c|.
  Veja as opções do programa e execute cada uma delas. É necessário ser
  usuário \verb|root| para alguma delas.

  \begin{solution}
    \begin{Verbatim}[label={\$ ./prog1 d}]
    main()                          Testing Default Scheduling
    Falha de segmentação (imagem do núcleo gravada)
    \end{Verbatim}

    \begin{Verbatim}[label={\$ sudo ./prog1 r}, fontsize=\small]
    main()                          Testing Round-Robin Scheduling
    main()                          setting priority(1-1) of thread 0 to 1
    main()                          setting priority(1-1) of thread 1 to 1
    
    work_routine()  thread 0        I'm Alive
    main()                          setting priority(1-1) of thread 2 to 1
    
    work_routine()  thread 1        I'm Alive
    
    main()                          3 threads created. Main running fifo at max
    
    work_routine()  thread 2        I'm Alive
    
    work_routine()  thread 2        I'm Proceeding
    print_count()   thread 2                          78
    work_routine()  thread 0        I'm Proceeding
    print_count()   thread 2           6              82
    work_routine()  thread 1        I'm Proceeding
    print_count()   thread 1        96778   99998   99794
    work_routine()  thread 1        my count 100000, total_count 100000
    print_count()   thread 0        97038    196    99998
    work_routine()  thread 2        my count 100000, total_count 200000
    print_count()   thread 1        99998   3394    2982
    [...]
    work_routine()  thread 1        my count 400000, total_count 1000000
    print_count()   thread 2        99998   4096    91806
    work_routine()  thread 0        my count 400000, total_count 1000000
    
    workroutine()   thread 0        I'm done
    print_count()   thread 2                4130    91856
    main()                          joined to thread 0 
    print_count()   thread 1                12246   99998
    work_routine()  thread 2        my count 400000, total_count 1000000
    
    workroutine()   thread 2        I'm done
    print_count()   thread 1                99998
    work_routine()  thread 1        my count 500000, total_count 1000000
    
    workroutine()   thread 1        I'm done
    
    main()                          joined to thread 1 
    
    main()                          joined to thread 2 
    main()                          all 3 threads have finished. 
    main()                          total_count 1000000
    main()                          thread 0, count 400000 percent 0.400000
    main()                          thread 1, count 500000 percent 0.500000
    main()                          thread 2, count 400000 percent 0.400000
    \end{Verbatim}
    \vspace{0.1em}
    \begin{Verbatim}[label={\$ sudo ./prog1 t}, fontsize=\small]
    main()                          Testing Round-Robin/Priority Scheduling
    main()                          setting priority(1-6) of thread 0 to 1
    main()                          setting priority(1-6) of thread 1 to 3
    
    work_routine()  thread 0        I'm Alive
    main()                          setting priority(1-6) of thread 2 to 6
    
    work_routine()  thread 1        I'm Alive
    
    main()                          3 threads created. Main running fifo at max
    
    work_routine()  thread 2        I'm Alive
    
    work_routine()  thread 2        I'm Proceeding
    print_count()   thread 2                           0
    work_routine()  thread 1        I'm Proceeding
    print_count()   thread 1                  10       4
    work_routine()  thread 0        I'm Proceeding
    print_count()   thread 0        77018   89026   99998
    work_routine()  thread 2        my count 100000, total_count 100000
    print_count()   thread 2        87254   99998   13400
    work_routine()  thread 1        my count 100000, total_count 200000
    print_count()   thread 0        99998   14450   30414
    [...]
    work_routine()  thread 2        my count 400000, total_count 1000000
    print_count()   thread 2        56346   99998   63620
    work_routine()  thread 1        my count 400000, total_count 1000000
    
    workroutine()   thread 1        I'm done
    print_count()   thread 2        92910           99998
    work_routine()  thread 2        my count 500000, total_count 1000000
    
    workroutine()   thread 2        I'm done
    print_count()   thread 0        99998
    work_routine()  thread 0        my count 400000, total_count 1000000
    
    workroutine()   thread 0        I'm done
    
    main()                          joined to thread 0 
    
    main()                          joined to thread 1 
    
    main()                          joined to thread 2 
    main()                          all 3 threads have finished. 
    main()                          total_count 1000000
    main()                          thread 0, count 400000 percent 0.400000
    main()                          thread 1, count 400000 percent 0.400000
    main()                          thread 2, count 500000 percent 0.500000
    \end{Verbatim}
    \vspace{0.1em}
    \begin{Verbatim}[label={\$ sudo ./prog1 f}, fontsize=\small]
    main()                          Testing FIFO Scheduling
    main()                          setting priority(1-1) of thread 0 to 1
    main()                          setting priority(1-1) of thread 1 to 1
    
    work_routine()  thread 0        I'm Alive
    main()                          setting priority(1-1) of thread 2 to 1
    
    work_routine()  thread 1        I'm Alive
    
    main()                          3 threads created. Main running fifo at max
    
    work_routine()  thread 2        I'm Alive
    
    work_routine()  thread 2        I'm Proceeding
    print_count()   thread 2                           2
    work_routine()  thread 1        I'm Proceeding
    print_count()   thread 1                  16       8
    work_routine()  thread 0        I'm Proceeding
    print_count()   thread 2        97568   97726   99998
    work_routine()  thread 2        my count 100000, total_count 100000
    print_count()   thread 2        99392   99998   2420
    work_routine()  thread 1        my count 100000, total_count 200000
    print_count()   thread 0        99998    728    3066
    [...]
    work_routine()  thread 1        my count 400000, total_count 1000000
    print_count()   thread 0        97482    648    99998
    work_routine()  thread 2        my count 400000, total_count 1000000
    
    workroutine()   thread 2        I'm done
    print_count()   thread 0        99998   3272
    work_routine()  thread 0        my count 400000, total_count 1000000
    
    workroutine()   thread 0        I'm done
    print_count()   thread 1                3364
    main()                          joined to thread 0 
    print_count()   thread 1                99998
    work_routine()  thread 1        my count 500000, total_count 1000000
    
    workroutine()   thread 1        I'm done
    
    main()                          joined to thread 1 
    
    main()                          joined to thread 2 
    main()                          all 3 threads have finished. 
    main()                          total_count 1000000
    main()                          thread 0, count 400000 percent 0.400000
    main()                          thread 1, count 500000 percent 0.500000
    main()                          thread 2, count 400000 percent 0.400000
    \end{Verbatim}
    \vspace{0.1em}
    \begin{Verbatim}[label={\$ sudo ./prog1 p}, fontsize=\small]
    main()                          Testing FIFO/Priority Scheduling
    main()                          setting priority(1-6) of thread 0 to 1
    main()                          setting priority(1-6) of thread 1 to 3
    
    work_routine()  thread 0        I'm Alive
    main()                          setting priority(1-6) of thread 2 to 6
    
    work_routine()  thread 1        I'm Alive
    
    main()                          3 threads created. Main running fifo at max
    
    work_routine()  thread 2        I'm Alive
    
    work_routine()  thread 2        I'm Proceeding
    print_count()   thread 2                          16
    work_routine()  thread 1        I'm Proceeding
    print_count()   thread 2                 420      18
    work_routine()  thread 0        I'm Proceeding
    print_count()   thread 1        80862   96224   99998
    work_routine()  thread 2        my count 100000, total_count 100000
    print_count()   thread 2        84170   99998   3104
    work_routine()  thread 1        my count 100000, total_count 200000
    print_count()   thread 0        99998   18118   23976
    [...]
    work_routine()  thread 2        my count 400000, total_count 1000000
    print_count()   thread 1        42636   99998   39172
    work_routine()  thread 1        my count 400000, total_count 1000000
    
    workroutine()   thread 1        I'm done
    print_count()   thread 2        99998           98594
    work_routine()  thread 0        my count 400000, total_count 1000000
    
    workroutine()   thread 0        I'm done
    print_count()   thread 2                        98678
    main()                          joined to thread 0 
    
    main()                          joined to thread 1 
    print_count()   thread 2                        99998
    work_routine()  thread 2        my count 500000, total_count 1000000
    
    workroutine()   thread 2        I'm done
    
    main()                          joined to thread 2 
    main()                          all 3 threads have finished. 
    main()                          total_count 1000000
    main()                          thread 0, count 400000 percent 0.400000
    main()                          thread 1, count 400000 percent 0.400000
    main()                          thread 2, count 500000 percent 0.500000  
    \end{Verbatim}
  \end{solution} 
\end{parts}

\question
Uso de variáveis de condição.

\begin{parts}
  \part
  Compile com a opção \verb|-pthread| e execute o programa \verb|prog2.c|.
  Explique o uso das variáveis de condição no código do programa.

  \begin{solution}
    Variáveis de condição são interessantes para indicar uma ocorrência
    de semáforos etc. Para evitar condições de corrida, usa-se o 
    \emph{Mutex} do \emph{Pthread}, com variáveis de condição, implementando
    um método de exclusão mútua, permitindo proteger tais áreas. A sinalização 
    entre os dois \emph{threads}, de produtor e consumidor, é feito através 
    das variáveis de condição, bloqueando e esperando ser acordado. Quando o 
    \emph{buffer} do produtor enche, o produtor é bloqueado e sinaliza o 
    consumidor para  consumir os itens. Ao consumir todos os itens e esvaziar
    o \emph{buffer}, o consumidor é bloqueado e sinaliza o produtor para
    produzir mais itens.

    \begin{Verbatim}[label={\$ ./prog2}]
    Consumidor iniciando...
    Produtor iniciando...
    Buffer vazio -- consumidor aguardando...
    item 0 inserido (valor=77)
    item 1 inserido (valor=15)
    item 2 inserido (valor=93)
    item 3 inserido (valor=35)
    item 4 inserido (valor=86)
    item 5 inserido (valor=92)
    item 6 inserido (valor=21)
    item 7 inserido (valor=27)
    Buffer cheio -- produtor aguardando...
    Consumidor reiniciando...
    item 7 removido (valor=27)
    item 6 removido (valor=21)
    item 5 removido (valor=92)
    item 4 removido (valor=86)
    item 3 removido (valor=35)
    item 2 removido (valor=93)
    item 1 removido (valor=15)
    item 0 removido (valor=77)
    Buffer vazio -- consumidor aguardando...
    Produtor reiniciando...
    [...]
    \end{Verbatim}
  \end{solution}
\end{parts}

\question
Impasse.

\begin{parts}
  \part
  Compile com a opção \verb|-pthread| e execute o programa \verb|prog3.c|.
  O que aconteceu na saída? Justifique.

  \begin{solution}
    Os filósofos pegam inicialmente o garfo da esquerda, não
    conseguindo pegar o da direita posteriormente, gerando
    um \emph{deadlock}. Isto acontece pois na hora que todos pegam
    o garfo esquerdo, não há a possibilidade de pegar o garfo
    direito já que está sendo usado pelo filósofo ao lado.

    \begin{Verbatim}[label={\$ ./prog3}]
    - H - T - T - T - T - T - T - T - T - T -
    - H - H - T - T - T - T - T - T - T - T -
    - H - H - H - T - T - T - T - T - T - T -
    - H - H - H - H - T - T - T - T - T - T -
    - H - H - H - H - T - T - H - T - T - T -
    - H - H - H - H - T - T - H - H - T - T -
    - H - H - H - H - T - T - H - H - H - T -
    - H - H - H - H - T - H - H - H - H - T -
    - H - H - H - H - T - H - H - H - H - H -
    - H - H - H - H - H - H - H - H - H - H -
     |H - H - H - H - H - H - H - H - H - H  
     |H - H  |H - H - H - H - H - H - H - H  
     |H - H  |H  |H - H - H - H - H - H - H  
     |H  |H  |H  |H - H - H - H - H - H - H  
     |H  |H  |H  |H - H - H - H  |H - H - H  
     |H  |H  |H  |H - H - H  |H  |H - H - H  
     |H  |H  |H  |H - H - H  |H  |H - H  |H  
     |H  |H  |H  |H - H - H  |H  |H  |H  |H  
     |H  |H  |H  |H - H  |H  |H  |H  |H  |H  
     |H  |H  |H  |H  |H  |H  |H  |H  |H  |H 
    \end{Verbatim}
  \end{solution}

  \part
  Compile com a opção \verb|-pthread| e execute o programa \verb|prog4.c|,
  que é baseado na Figura 2.38 do livro texto. O problema do programa
  anterior foi resolvido? Justifique.

  \begin{solution}
    Sim, o problema do programa anterior foi resolvido evitando
    condições de corrida através de dois semáforos, onde quando
    há o \emph{down} no \emph{mutex}, garante exclusivamente os 
    garfos para o filósofo. Há também um semáforo exclusivo para 
    cada filósofo para evitar a mudança de estados enquanto seus 
    vizinhos tiverem garfos em mão.

    \begin{Verbatim}[label={\$ ./prog4}]
    T T T T T T T T T T T 
    H T T T T T T T T T T 
    E T T T T T T T T T T 
    E T T T T H T T T T T 
    E T T T T E T T T T T 
    E T T H T E T T T T T 
    E T T E T E T T T T T 
    E T T E T E T H T T T 
    E T T E T E T E T T T 
    E T T E T E H E T T T 
    E T T E H E H E T T T 
    [...]
    Total de refeicoes: 
    Filosofo 0: 10
    Filosofo 1: 9
    Filosofo 2: 9
    Filosofo 3: 9
    Filosofo 4: 9
    Filosofo 5: 9
    Filosofo 6: 9
    Filosofo 7: 9
    Filosofo 8: 9
    Filosofo 9: 9
    Filosofo 10: 9
    \end{Verbatim}
  \end{solution}
\end{parts}

\question
Inanição (\emph{starvation}) e desempenho.

\begin{parts}
  \part
  Compile com a opção \verb|-pthread| e execute o programa \verb|prog5.c|.
  O que aconteceu na saída? Justifique.

  \begin{solution}
    Os leitores tentam inicialmente ler seu valor na base de dados,
    bloqueando a escrita e leitura na base, onde somente quando todos
    os leitores tiverem usado a base os escritores serão permitidos
    para escrever tais valores, fazendo com que os escritores sofram
    de inanição. Em resumo, os escritores só conseguem entrar se
    não tiver nenhum leitor ativo no momento.

    \begin{Verbatim}[label={\$ time ./prog5}]
    Leitor 0 leu o valor -1 
    Leitor 4 leu o valor -1 
    Leitor 1 leu o valor -1 
    Leitor 2 leu o valor -1 
    [...]
    Leitor 146 leu o valor -1 
    Escritor 0 escreveu o valor 0 
    Escritor 1 escreveu o valor 1 
    Escritor 3 escreveu o valor 3 
    Escritor 4 escreveu o valor 4 
    Escritor 5 escreveu o valor 5 
    [...]

    real    2m34.030s
    user    0m0.008s
    sys     0m0.008s
    \end{Verbatim}
  \end{solution}

  \part
  Compile com a opção \verb|-pthread| e execute o programa \verb|prog6.c|.
  O problema do programa anterior foi resolvido? Há espaço para melhoria
  de desempenho da leitura no código?

  \begin{solution}
    A inanição foi resolvida, no entanto o paralelismo da leitura
    foi perdido, ainda não resolvendo o problema de desempenho.
    Leitores e escritores estão competindo de forma aproximadamente
    equalitária.

    \begin{Verbatim}[label={\$ time ./prog6}]
    Leitor 0 leu o valor -1 
    Escritor 0 escreveu o valor 0 
    Leitor 1 leu o valor 0 
    Leitor 2 leu o valor 0 
    Leitor 3 leu o valor 0 
    Leitor 4 leu o valor 0 
    Escritor 1 escreveu o valor 1 
    Escritor 2 escreveu o valor 2 
    Leitor 5 leu o valor 2 
    Leitor 6 leu o valor 2 
    Escritor 3 escreveu o valor 3 
    Leitor 7 leu o valor 3 
    Escritor 4 escreveu o valor 4 
    Leitor 8 leu o valor 4 
    Leitor 9 leu o valor 4 
    Escritor 5 escreveu o valor 5
    [...]

    real    5m0.045s
    user    0m0.000s
    sys     0m0.020s
    \end{Verbatim}
  \end{solution}

  \part
  Compile com a opção \verb|-pthread| e execute o programa \verb|prog7.c|.
  Houve melhoria de desempenho na leitura em relação ao programa anterior?
  Justifique.

  \begin{solution}
    Sim, houve melhorias com a utilização de variáveis de condição
    que permitem que haja um paralelismo. Quando não há mais nenhum
    leitor, é sinalizado que os escritores, que tem acesso exclusivo,
    podem utilizar a base de dados. Desta maneira, os leitores
    mantém seu paralelismo, sendo a solução correta para o problema.

    \begin{Verbatim}[label={\$ time ./prog7}]
    Leitor 0 leu o valor -1 
    Leitor 2 leu o valor -1 
    Leitor 3 leu o valor -1 
    Leitor 4 leu o valor -1 
    Escritor 1 escreveu o valor 1 
    Escritor 2 escreveu o valor 2 
    Leitor 5 leu o valor 2 
    Leitor 6 leu o valor 2 
    Escritor 3 escreveu o valor 3 
    Escritor 4 escreveu o valor 4
    [...]

    real    3m49.038s
    user    0m0.012s
    sys     0m0.012s
    \end{Verbatim}
  \end{solution}
\end{parts}