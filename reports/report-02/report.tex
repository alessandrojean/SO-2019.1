% !TeX program = xelatex
% !TeX encoding = UTF-8
\documentclass[a4paper, 11pt, answers]{exam}

\usepackage[brazil]{babel}
\let\latinencoding\relax
\usepackage[T1]{fontenc}

\usepackage[no-math]{fontspec}
\usepackage{newpxtext, newpxmath}
\usepackage[a4paper, margin=2cm]{geometry}

\usepackage{siunitx}
\usepackage{fancyvrb}
\usepackage{graphicx}

\input{definition.tex}

\title{\titulo}
\author{\nomeAutorUm e \nomeAutorDois}
\date{\today}

\footer{}{}{\thepage}

\sisetup{
  binary-units = true
}

\renewcommand{\solutiontitle}{}

\fvset{
  gobble = 8,
  frame = single
}

\newcommand{\printtitle}{
  \begin{center}
    {\Large \scshape \titulo}\\[1em]
    {\nomeAutorUm, \raAutorUm}\\
    {\nomeAutorDois, \raAutorDois}\\[1em]
    Professor: Dr\@. \nomeProfessor, \centroProfessor\\
    {\itshape \campusFaculdade}
  \end{center}
}

\begin{document}
  \printtitle

  \begin{questions}
    \question
    Algoritmos para alocação de memória livre.

    \begin{parts}
      \part
      Compile e execute o programa \verb|prog1.c|. Que algoritmo de
      alocação de memória livre foi usado para \verb|s5| e \verb|s6|?
      Justifique.

      \begin{solution}
        A estratégia interna que o sistema de alocação de memória livre usa
        dentro do espaço de endereçamento do processo é o \emph{best-fit},
        já que utiliza-se a menor lacuna disponível na memória livre para
        o alocamento das novas variáveis do programa.

        \begin{Verbatim}[label={\$ ./prog1}]
        Size of int: 4 bytes
        Aloca s1: 0xd0d420
        Aloca s2: 0xd0d5c0
        Aloca s3: 0xd0d690
        Aloca s4: 0xd0d790
        Libera s1.
        Libera s3.
        Aloca s5: 0xd0d690
        Aloca s6: 0xd0d6b0
        \end{Verbatim}
      \end{solution}
    \end{parts}

    \question
    Descobrindo o tamanho de página.

    \begin{parts}
      \part
      Compile e execute o programa \verb|prog2.c| com a \emph{flag}
      \verb|-lm| após \verb|prog2.c|. Qual o tamanho de página do
      sistema obtida pela chamada de sistema \verb|getpagesize|?

      \begin{solution}
        O tamanho de página do sistema obtida pela chamada de sistema
        é de \SI{4}{\kilo\byte}, ou \SI{4096}{\byte}, tendo \SI{12}{\bit}.
        O tamanho máximo do espaço de endereçamento é de \SI{64}{\bit}.
        
        \begin{Verbatim}[label={\$ ./prog2}]
        Endereço de global: 0x60104c
        Tamanho de uma página 0x1000, 12 bits
        Endereço: 8 bytes, 64 bits
        \end{Verbatim}
      \end{solution}

      \part
      Instale o pacote \verb|hugepages| e descubra o tamanho das páginas
      suportadas com o comando \verb|pagesize -a|. Por que é importante
      suportar outros tamanhos de página, além do padrão?

      \begin{solution}
        A vantagem de suportar páginas maiores é diminuir a quantidade
        de páginas, tornando mais barato trabalhar com um número
        menor de páginas; a tabela fica menor, sendo mais fácil
        de ser varrida, facilitando o gerenciamento e aumentando
        as chances de ter um \emph{cache heap} na TLB.

        \begin{Verbatim}[label={\$ pagesize -a}]
        4096
        2097152
        \end{Verbatim}
      \end{solution}
    \end{parts}

    \question
    Alterando o tamanho do segmento de dados.

    \begin{parts}
      \part
      Compile e execute o programa \verb|prog3.c|. Remova o comentário
      abaixo de ``Acarretaria \linebreak SEGMENTATION\_FAULT'' e recompile e reexecute
      o programa. Por que ocorreu um \emph{segmentation fault}?

      \begin{solution}
        Ocorre um \emph{segmentation fault} pois aumenta-se um número
        menor do que uma página, então acaba-se acessando a próxima
        página, gerando a falha. Como retorna-se uma página inteira
        imediatamente antes, acaba-se acessando a próxima página,
        gerando a falha. O segundo caso não ocasiona falhas pois está
        no meio da página, não causando que o byte fique fora da
        área de dados. O mínimo a ser alocado é uma página, não há
        como alocar, por exemplo, $1/4$ de página.

        \begin{Verbatim}[label={\$ ./prog3}]
        Tamanho em bytes de um endereço: 8
        Tamanho em bytes de uma página: 4096
        str_pilha  = 0x7ffe1729a08e
        main       = 0x400666
        str_dados  = 0x601060
        Topo da área de dados = 0x1e30000
        Topo da área de dados = 0x1e31000
        Topo da área de dados = 0x1e31400          
        \end{Verbatim}

        \begin{Verbatim}[label={\$ ./prog3}]
        Tamanho em bytes de um endereço: 8
        Tamanho em bytes de uma página: 4096
        str_pilha  = 0x7fffcf39e30e
        main       = 0x400666
        str_dados  = 0x601060
        Topo da área de dados = 0x225e000
        Topo da área de dados = 0x225f000
        Falha de segmentação (imagem do núcleo gravada)       
        \end{Verbatim}
      \end{solution}
    \end{parts}

    \question
    Memória compartilhada.

    \begin{parts}
      \part
      Compile e execute o programa \verb|prog4.c|. Examine o conteúdo do
      arquivo \verb|teste_mmap.txt|. Quais as vantagens e desvantagens de
      se mapear um arquivo em memória?

      \begin{solution}
        O conteúdo do arquivo contém a frase, no entanto, como mapeia-se
        \SI{100}{\byte} da memória, o resto do arquivo são resíduos. Pode-se
        utilizar este arquivo para comunicar-se entre processos. A leitura
        do arquivo em memória é muito mais rápida. No entanto, a complexidade
        é maior e pode possibilitar o acontecimento de condições de corrida.
        Pode ocorrer um problema de falha onde há atualização na memória
        mas não em disco também.

        \begin{Verbatim}[label={\$ sudo less teste\_mmap.txt}]
        Testando mmap novamente.
        \end{Verbatim}
      \end{solution}

      \part
      Compile e execute o programa \verb|prog5.c|. O que aconteceu com o
      conteúdo do arquivo \verb|teste_mmap.txt|?

      \begin{solution}
        Troca-se uma letra do arquivo de saída, o \verb|teste_mmap.txt|.
        Em tese, pode-se ter mais de um processo rodando e todos
        podem estar escrevendo no mesmo arquivo. No entanto, torna-se
        necessário criar um mecanismo de exclusão mútua para evitar
        que um sobreescreva o outro.

        \begin{Verbatim}[label={\$ sudo less teste\_mmap.txt}]
        Teztando mmap novamente.
        \end{Verbatim}
      \end{solution}
    \end{parts}

    \question
    Verificando a quantidade de faltas de páginas.

    \begin{parts}
      \part
      Compile e execute o programa \verb|prog6.c|. O laço usado na função
      \verb|preenche_lento| é a forma mais eficiente de iteração sobre
      uma matriz? Justifique.

      \begin{solution}
        O programa possui uma complexidade quadrática, realizando
        $3.2$ bilhões de operações. Portanto, o laço usado na função
        não é a forma mais eficiente de iteração sobre uma matriz.
        Como a matriz é varrida no sentido errado, coluna em vez de linha
        a cada iteração, lê-se uma página toda vez, já que cada
        linha é uma página. Desta maneira, há um \emph{miss} na TLB e no
        \emph{cache} do processador.

        \begin{Verbatim}[label={\$ /usr/bin/time -v ./prog6}]
        Topo da área de dados = 0xc4ca2000
        Inicio preenche lento.
        Fim preenche lento.
        Falhas de página sem ida ao disco: 800058
        Falhas de página com ida ao disco: 0

                Command being timed: "./prog6"
                User time (seconds): 47.43
                System time (seconds): 0.68
                Percent of CPU this job got: 99%
                Elapsed (wall clock) time (h:mm:ss or m:ss): 0:48.12
                Average shared text size (kbytes): 0
                Average unshared data size (kbytes): 0
                Average stack size (kbytes): 0
                Average total size (kbytes): 0
                Maximum resident set size (kbytes): 3201332
                Average resident set size (kbytes): 0
                Major (requiring I/O) page faults: 0
                Minor (reclaiming a frame) page faults: 800059
                Voluntary context switches: 1
                Involuntary context switches: 81
                Swaps: 0
                File system inputs: 0
                File system outputs: 0
                Socket messages sent: 0
                Socket messages received: 0
                Signals delivered: 0
                Page size (bytes): 4096
                Exit status: 0
        \end{Verbatim}
      \end{solution}

      \part
      Modifique o programa \verb|prog6.c| para aumentar o desempenho
      do laço. Houve redução no número de faltas de página em relação
      ao programa anterior? Justifique.

      \begin{solution}
        Não houve mudança no número de faltas de página, já que só
        há uma mudança de impacto na TLB, em relação ao processador
        e ao \emph{cache}. A diferença é que quando acessa-se um elemento
        na mesma página, há um \emph{hit} na TLB, atingindo o \emph{cache}
        L1 do processador, gerando essa diferença brutal em questões
        de processamento e desempenho.

        \begin{Verbatim}[label={\$ /usr/bin/time -v ./prog6}]
        Topo da área de dados = 0xc4071000
        Inicio preenche lento.
        Fim preenche lento.
        Falhas de página sem ida ao disco: 800058
        Falhas de página com ida ao disco: 0
      
                Command being timed: "./prog6-e"
                User time (seconds): 6.84
                System time (seconds): 0.61
                Percent of CPU this job got: 99%
                Elapsed (wall clock) time (h:mm:ss or m:ss): 0:07.45
                Average shared text size (kbytes): 0
                Average unshared data size (kbytes): 0
                Average stack size (kbytes): 0
                Average total size (kbytes): 0
                Maximum resident set size (kbytes): 3201304
                Average resident set size (kbytes): 0
                Major (requiring I/O) page faults: 0
                Minor (reclaiming a frame) page faults: 800059
                Voluntary context switches: 1
                Involuntary context switches: 12
                Swaps: 0
                File system inputs: 0
                File system outputs: 0
                Socket messages sent: 0
                Socket messages received: 0
                Signals delivered: 0
                Page size (bytes): 4096
                Exit status: 0
        \end{Verbatim}
      \end{solution}
    \end{parts}

    \question
    Estatísticas.

    \begin{parts}
      \part
      Use o comando \verb|ps -o min_flt,maj_flt,cmd PID| para enxergar
      as estatísticas de falta de páginas de um processo \verb|PID|.

      \begin{solution}
        Utilizou-se como exemplo o \verb|PID| do Firefox recém-aberto.

        \begin{Verbatim}[label={\$ ps -o min\_flt,maj\_flt,cmd 10628}]
         MINFL  MAJFL CMD
        151024      0 /usr/lib/firefox/firefox
        \end{Verbatim}
      \end{solution}

      \part
      Use o comando \verb|top|, tecle \verb|F| e selecione as faltas
      de página.

      \begin{solution}
        Selecionou-se no menu as opções \verb|nMaj| e \verb|nMin|.

        \begin{Verbatim}[label={\$ top}, fontsize=\scriptsize]
        top - 09:52:41 up  2:31,  1 user,  load average: 0,36, 0,25, 0,25
        Tarefas: 220 total,   1 executando, 219 dormindo,   0 parado,   0 zumbi
        %Cpu(s):  0,9 us,  0,3 sy,  0,0 ni, 97,5 id,  1,3 wa,  0,0 hi,  0,0 si,  0,0 st
        KiB Mem :  8058996 total,  3070176 free,  1448752 used,  3540068 buff/cache
        KiB Swap:  8270844 total,  8261140 free,     9704 used.  5804584 avail Mem 
        
          PID USUÁRIO  PR  NI    VIRT    RES    SHR S  %CPU     TIME+ COMMAND        %MEM nMaj nMin 
          927 root      20   0  600180 127036  97172 S   3,3   2:15.24 Xorg            1,6   81 135k 
         8043 ufabc     20   0  668176  39364  29984 S   2,0   0:02.57 gnome-termina+  0,5    0 5653 
         1963 ufabc     20   0 1675732 233112  66148 S   1,7   1:46.08 compiz          2,9  170 147k 
         7459 ufabc     20   0 1982960 304056  90920 S   0,3   4:06.66 code            3,8    0 2,4m 
        10147 root      20   0       0      0      0 S   0,3   0:00.53 kworker/u8:0    0,0    0    0 
        10593 ufabc     20   0   41872   3832   3204 R   0,3   0:00.50 top             0,0    2  304 
        10752 ufabc     20   0 1532996 108004  82784 S   0,3   0:00.75 WebExtensions   1,3    0  14k 
            1 root      20   0  119892   5740   4080 S   0,0   0:02.24 systemd         0,1   54  35k 
            2 root      20   0       0      0      0 S   0,0   0:00.00 kthreadd        0,0    0    0 
            3 root      20   0       0      0      0 S   0,0   0:00.02 ksoftirqd/0     0,0    0    0 
            5 root       0 -20       0      0      0 S   0,0   0:00.00 kworker/0:0H    0,0    0    0 
            7 root      20   0       0      0      0 S   0,0   0:06.86 rcu_sched       0,0    0    0 
            8 root      20   0       0      0      0 S   0,0   0:00.00 rcu_bh          0,0    0    0 
            9 root      rt   0       0      0      0 S   0,0   0:00.00 migration/0     0,0    0    0 
           10 root      rt   0       0      0      0 S   0,0   0:00.02 watchdog/0      0,0    0    0 
           11 root      rt   0       0      0      0 S   0,0   0:00.02 watchdog/1      0,0    0    0 
           12 root      rt   0       0      0      0 S   0,0   0:00.00 migration/1     0,0    0    0 
        \end{Verbatim}
      \end{solution}

      \part
      Use o comando \verb|/usr/bin/time -v CMD| para obter estatísticas de um
      programa executado \verb|CMD|.

      \begin{solution}
        Como exemplo, optou-se por utilizar o comando \verb|ls|.

        \begin{Verbatim}[label={\$ /usr/bin/time -v ls}, fontsize=\small]
        prog1    prog2    prog3    prog4    prog5    prog6    prog6-e    README.md
        prog1.c  prog2.c  prog3.c  prog4.c  prog5.c  prog6.c  prog6-e.c  teste_mmap.txt

                Command being timed: "ls"
                User time (seconds): 0.00
                System time (seconds): 0.00
                Percent of CPU this job got: 0%
                Elapsed (wall clock) time (h:mm:ss or m:ss): 0:00.00
                Average shared text size (kbytes): 0
                Average unshared data size (kbytes): 0
                Average stack size (kbytes): 0
                Average total size (kbytes): 0
                Maximum resident set size (kbytes): 2412
                Average resident set size (kbytes): 0
                Major (requiring I/O) page faults: 0
                Minor (reclaiming a frame) page faults: 108
                Voluntary context switches: 1
                Involuntary context switches: 1
                Swaps: 0
                File system inputs: 0
                File system outputs: 0
                Socket messages sent: 0
                Socket messages received: 0
                Signals delivered: 0
                Page size (bytes): 4096
                Exit status: 0
        \end{Verbatim}
      \end{solution}
    \end{parts}
  \end{questions}
\end{document}
