% !TeX program = xelatex
% !TeX encoding = UTF-8
\documentclass[a4paper, 11pt, answers]{exam}

\usepackage[brazil]{babel}
\let\latinencoding\relax
\usepackage[T1]{fontenc}

\usepackage[no-math]{fontspec}
\usepackage{newpxtext, newpxmath}
\usepackage[a4paper, margin=2cm]{geometry}

\usepackage{siunitx}
\usepackage{fancyvrb}
\usepackage{graphicx}

\input{definition.tex}

\title{\titulo}
\author{\nomeAutorUm e \nomeAutorDois}
\date{\today}

\footer{}{}{\thepage}

\sisetup{
  binary-units = true
}

\renewcommand{\solutiontitle}{}

\fvset{
  gobble = 8,
  frame = single
}

\newcommand{\printtitle}{
  \begin{center}
    {\Large \scshape \titulo}\\[1em]
    {\nomeAutorUm, \raAutorUm}\\
    {\nomeAutorDois, \raAutorDois}\\[1em]
    Professor: Dr\@. \nomeProfessor, \centroProfessor\\
    {\itshape \campusFaculdade}
  \end{center}
}

\begin{document}
  \printtitle

  \begin{questions}
    \question
    Leitura e escrita em arquivos.

    \begin{parts}
      \part
      Compile e execute o programa \verb|prog1.c|. Digite sequências de
      caracteres no terminal seguido de \verb|ENTER|. O que o programa
      faz? Por que ele não tem \verb|include|? Por que não é preciso abrir
      os descritores da entrada e saída padrão?

      \begin{solution}
        O programa pega a entrada do usuário da entrada padrão
        e a escreve na tela utilizando a saída padrão. Não há a 
        existência de \verb|include| pois o código realiza chamadas
        de sistema, não usando funções de bibliotecas, que já estão 
        disponíveis pelo sistema operacional. Não é preciso abrir os 
        descritores pois a entrada e saída padrão são abertos por definição.

        \begin{Verbatim}[label={\$ ./prog1}]
        Este   
        Este
        eh
        eh
        um
        um
        teste
        teste
        \end{Verbatim}
      \end{solution}
    \end{parts}

    \question
    Remoção de arquivos.

    \begin{parts}
      \part
      Compile e execute o programa \verb|prog2.c|. Digite uma sequência de
      caracteres seguido de \verb|ENTER|. O que acontece com o arquivo
      \verb|temp.txt| no diretório corrente?

      \begin{solution}
        O arquivo recebe o conteúdo que foi digitado. Agora utiliza-se
        funções de biblioteca para a manipulação da entrada do usuário
        e para a escrita no arquivo \verb|temp.txt|. Utiliza-se também
        \emph{flags} para determinação de leitura e escrita no arquivo.
        O programa dá as opções de deletar o arquivo, através de
        \verb|unlink|, escrever no arquivo ou ler seu conteúdo.
        Quando o programa é encerrado, ao executar novamente, por
        causa da \emph{flag truncate}, seu conteúdo será apagado.
        
        \begin{Verbatim}[label={\$ ./prog2}]
        Digite unlink para chamada unlink, print para imprimir conteúdo e quit 
        para encerrar o programa: este
        Grava e retorna ao loop apos comando: este 
        Digite unlink para chamada unlink, print para imprimir conteúdo e quit 
        para encerrar o programa: eh
        Grava e retorna ao loop apos comando: eh 
        Digite unlink para chamada unlink, print para imprimir conteúdo e quit 
        para encerrar o programa: um
        Grava e retorna ao loop apos comando: um 
        Digite unlink para chamada unlink, print para imprimir conteúdo e quit 
        para encerrar o programa: teste
        Grava e retorna ao loop apos comando: teste 
        Digite unlink para chamada unlink, print para imprimir conteúdo e quit 
        para encerrar o programa: print
        este
        eh
        um
        teste
        Digite unlink para chamada unlink, print para imprimir conteúdo e quit 
        para encerrar o programa: quit
        \end{Verbatim}
      \end{solution}

      \part
      Crie um \emph{hard link} do arquivo \verb|temp.txt| com o comando
      \verb|ln temp.txt clone.txt|. Em seguida, entre com o comando
      \verb|ls -i *.txt|. \verb|temp.txt| e \verb|clone.txt| são o 
      mesmo arquivo? Justifique.

      \begin{solution}
        Ambos são o mesmo arquivo já que possuem o mesmo número
        de identificação pelo \emph{i-node}. Como ambos possuem o
        mesmo número (\verb|5014739|), são o mesmo arquivo.

        \begin{Verbatim}[label={\$ ls -i *.txt}]
        5014739 clone.txt*  5014739 temp.txt*
        \end{Verbatim}
      \end{solution}

      \part
      Digite o comando \verb|unlink| na entrada de \verb|prog2.c|. 
      O que aconteceu? O arquivo \verb|clone.txt| ainda existe? Justifique.

      \begin{solution}
        O arquivo não foi removido pois ainda há uma referência do arquivo
        no \verb|clone.txt|. Como o contador não chegou em zero, já que ainda
        há um arquivo referenciando \verb|temp.txt|, então o arquivo não
        é removido.

        \begin{Verbatim}[label={\$ ./prog2}]
        Digite unlink para chamada unlink, print para imprimir conteúdo e quit 
        para encerrar o programa: unlink
        Unlink executado apos comando: unlink 
        Digite unlink para chamada unlink, print para imprimir conteúdo e quit 
        para encerrar o programa: quit
        \end{Verbatim}

      \end{solution}

      \part
      Digite uma sequência de caracteres seguido de \verb|ENTER|. O que
      acontece com o arquivo \verb|clone.txt| no diretório corrente? Por que
      é ainda possível ler o arquivo completo?

      \begin{solution}
        O arquivo \verb|clone.txt| é preenchido com os conteúdos digitados
        no programa ainda, mesmo que o original \verb|temp.txt| tenha sido
        removido. Ainda é possível ler o arquivo completo pois
        a chamada \verb|open| analisa o arquivo em si, mesmo que seu
        caminho tenha sido mudado.

        \begin{Verbatim}[label={\$ ./prog2}]
        Digite unlink para chamada unlink, print para imprimir conteúdo e quit 
        para encerrar o programa: teste
        Grava e retorna ao loop apos comando: teste 
        Digite unlink para chamada unlink, print para imprimir conteúdo e quit 
        para encerrar o programa: teste2
        Grava e retorna ao loop apos comando: teste2 
        Digite unlink para chamada unlink, print para imprimir conteúdo e quit 
        para encerrar o programa: unlink
        Unlink executado apos comando: unlink 
        Digite unlink para chamada unlink, print para imprimir conteúdo e quit 
        para encerrar o programa: teste3
        Grava e retorna ao loop apos comando: teste3 
        Digite unlink para chamada unlink, print para imprimir conteúdo e quit 
        para encerrar o programa: print
        teste
        teste2
        teste3
        Digite unlink para chamada unlink, print para imprimir conteúdo e quit 
        para encerrar o programa: quit
        \end{Verbatim}

      \end{solution}

      \part
      Em que outras situações é interessante esse comportamento do 
      \verb|unlink|?

      \begin{solution}
        É interessante utilizar o \verb|unlink| para arquivos temporários.
        Caso o processo morra, os arquivos temporários nas quais foram
        utilizados o \verb|unlink| são removidos do sistema.
      \end{solution}
    \end{parts}

    \question
    Uso de \emph{buffers}.

    \begin{parts}
      \part
      Compile e execute o programa \verb|prog3.c| com o seguinte comando
      \verb|/usr/bin/time -v CMD|, em que \verb|CMD| é o arquivo a ser
      executado. Verifique os gastos de tempo de CPU de saídas para o 
      sistema de arquivos, além do número de trocas de contexto voluntárias.

      \begin{solution}
        O programa possui execução rápida pois escreve-se os mil
        bytes em sequência, sem a utilização de \verb|fsync|. Como
        bloqueia-se apenas uma única vez, a execução torna-se
        rápida.

        \begin{Verbatim}[label={\$ time -v ./prog3}]
        Command being timed: "./prog3"
        User time (seconds): 0.00
        System time (seconds): 0.00
        Percent of CPU this job got: 0%
        Elapsed (wall clock) time (h:mm:ss or m:ss): 0:00.00
        Average shared text size (kbytes): 0
        Average unshared data size (kbytes): 0
        Average stack size (kbytes): 0
        Average total size (kbytes): 0
        Maximum resident set size (kbytes): 1172
        Average resident set size (kbytes): 0
        Major (requiring I/O) page faults: 0
        Minor (reclaiming a frame) page faults: 57
        Voluntary context switches: 1
        Involuntary context switches: 2
        Swaps: 0
        File system inputs: 0
        File system outputs: 8
        Socket messages sent: 0
        Socket messages received: 0
        Signals delivered: 0
        Page size (bytes): 4096
        Exit status: 0     
        \end{Verbatim}
      \end{solution}

      \part
      Compile e execute o programa \verb|prog4.c| com o seguinte comando
      \verb|/usr/bin/time -v CMD|, em que \verb|CMD| é o arquivo a ser 
      executado. Verifique os gastos de tempo de CPU de saídas para o 
      sistema de arquivos, além do número de trocas de contexto voluntárias.
      Por que o tempo da execução desse programa é muito maior do que
      o tempo de execução do programa anterior?

      \begin{solution}
        O programa passou mais tempo bloqueado do que fazendo processamento.
        A lentidão não é relacionada a falta de página. Este programa possui
        muitas mudanças de contexto. Cada vez que há uma chamada de 
        \verb|fsync| obriga-se o programa a realizar a saída, bloqueando-o 
        fazendo com que o escalonador escalone outro programa durante. Como 
        há mil chamadas de \verb|fsync|, o programa torna-se lento. Em 
        comparação ao \verb|prog3|, há somente um bloqueio pois escreve-se 
        apenas uma vez os mil bytes no disco.

        \begin{Verbatim}[label={\$ time -v ./prog4}]
        Command being timed: "./prog4"
        User time (seconds): 0.00
        System time (seconds): 0.11
        Percent of CPU this job got: 0%
        Elapsed (wall clock) time (h:mm:ss or m:ss): 0:36.08
        Average shared text size (kbytes): 0
        Average unshared data size (kbytes): 0
        Average stack size (kbytes): 0
        Average total size (kbytes): 0
        Maximum resident set size (kbytes): 1176
        Average resident set size (kbytes): 0
        Major (requiring I/O) page faults: 0
        Minor (reclaiming a frame) page faults: 57
        Voluntary context switches: 2002
        Involuntary context switches: 1
        Swaps: 0
        File system inputs: 0
        File system outputs: 8000
        Socket messages sent: 0
        Socket messages received: 0
        Signals delivered: 0
        Page size (bytes): 4096
        Exit status: 0   
        \end{Verbatim}
      \end{solution}
    \end{parts}

    \question
    Obtenção dos atributos de um arquivo.

    \begin{parts}
      \part
      Compile, ignorando os avisos da compilação, e execute o programa
      \verb|prog5.c| passando como parâmetro um arquivo qualquer do sistema
      de arquivos. Que informações ele me retorna? Guarde a saída da
      execução em diferentes tipos de arquivos para mostrar no relatório.

      \begin{solution}
        O programa mostra uma série de informações do arquivo tal como 
        seu tipo, usuário, grupo do usuário, permissão, tamanho,
        data de modificação, \emph{bits} de proteção etc.

        \begin{Verbatim}[label={\$ ./prog5 prog1.c}, fontsize=\small]
        prog1.c e' um arquivo regular 

        userid 1000, groupid 1000, permissao 100664, link(s) 1 
        tamanho 86 (bytes), 8 (blocos), Inode # 5006667 
        
        Dados de prog1.c foram modificados pela ultima vez em Wed Apr 17 08:08:14 2019
        
        Leitura permitida
        Escrita permitida
        Execucao proibida
        \end{Verbatim}

        \begin{Verbatim}[label={\$ ./prog5 .}, fontsize=\small]
        . e' um arquivo diretorio 

        userid 1000, groupid 1000, permissao 40775, link(s) 3 
        tamanho 4096 (bytes), 8 (blocos), Inode # 5003756 
        
        Dados de . foram modificados pela ultima vez em Wed Apr 17 09:12:17 2019
        
        Leitura permitida
        Escrita permitida
        Execucao permitida
        \end{Verbatim}
      \end{solution}
    \end{parts}

    \question
    Comando \verb|stat|.

    \begin{parts}
      \part
      O comando \verb|stat| tem função similar ao \verb|prog5.c|. Repita 
      a execução com esse comando, usando como entrada os mesmos arquivos do
      execício 4.1.

      \begin{solution}
        O comando retorna atributos e informações do arquivo como o 
        \verb|prog5|.

        \begin{Verbatim}[label={\$ stat prog1.c}, fontsize=\small]
          Arquivo: 'prog1.c'
          Tamanho: 86           Blocos: 8          Bloco IO: 4096   arquivo comum
        Dispositivo: 805h/2053d Inode: 5006667     Links: 1
        Acesso: (0664/-rw-rw-r--)  Uid: ( 1000/   ufabc)   Gid: ( 1000/   ufabc)
        Acessar: 2019-04-17 08:08:15.737457650 -0300
        Modificar: 2019-04-17 08:08:14.057482272 -0300
        Alterar: 2019-04-17 08:08:14.057482272 -0300
          Nascimento: -
        \end{Verbatim}

        \begin{Verbatim}[label={\$ stat .}, fontsize=\small]
          Arquivo: '.'
          Tamanho: 4096         Blocos: 8          Bloco IO: 4096   diretório
        Dispositivo: 805h/2053d Inode: 5003756     Links: 3
        Acesso: (0775/drwxrwxr-x)  Uid: ( 1000/   ufabc)   Gid: ( 1000/   ufabc)
        Acessar: 2019-04-17 09:12:17.697057433 -0300
        Modificar: 2019-04-17 09:12:17.681057243 -0300
        Alterar: 2019-04-17 09:12:17.681057243 -0300
          Nascimento: -
        \end{Verbatim}
      \end{solution}
    \end{parts}

    \question
    Lista de arquivos em um diretório.

    \begin{parts}
      \part
      Compile e execute o programa \verb|prog6.c| passando como parâmetro de
      entrada diversos diretórios do sistema de arquivo. Com base nas
      saídas, explique como funciona a organização dos arquivos em
      um diretório no Linux.

      \begin{solution}
        Um diretório no Linux possui diversas entradas que possuem informações
        como o número do \emph{i-node}, tamanho, nome e tipo. A organização, 
        resumidamente é uma lista que possui os nomes e o número de 
        \emph{i-node}, permitindo a estruturação do sistema de arquivos.

        \begin{Verbatim}[label={\$ ./prog6 .}]
        prog6.c (Inode 5007270)
        prog2.c (Inode 5007262)
        prog3.c (Inode 5007265)
        testeNoSync.txt (Inode 5010961)
        .. (Inode 4983190)
        answers.md (Inode 5010964)
        .vscode (Inode 5010945)
        . (Inode 5003756)
        prog5 (Inode 5014743)
        prog4.c (Inode 5007268)
        prog4 (Inode 5014735)
        prog1 (Inode 5007276)
        clone.txt (Inode 5010774)
        prog2 (Inode 5014736)
        prog5.c (Inode 5007269)
        prog6 (Inode 5015536)
        prog1.c (Inode 5006667)
        prog3 (Inode 5014738)
        testeSync.txt (Inode 5015535)
        \end{Verbatim}
      \end{solution}
    \end{parts}
  \end{questions}
\end{document}
